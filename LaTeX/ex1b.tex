\chapter*{Exercise 1b}

\section*{Thread load-balancing}
The code we present to this assignment is based on the 3rd example code for RTOS/ChibiOS. 
When uploaded to the Arduino $\mu$-Controller, the startup procedure is the same as in the example code. To enable dynamic load balancing, we added a call of \texttt{top\_load\_balance()} in the \texttt{static THD\_FUNCTION(top, arg)} function.
We implemented ou logic in \texttt{top\_load\_balance()}.

We read the thread-load of the last cycle and measure how close we are to the goal of 85\% CPU utilisation. We use se relative distance between \textsl{current load} to \textsl{target load} and compute a dynamic adjustment value. This value is added (or subtracted) to the value stored in \texttt{threadLoad[tid]}, where \texttt{tid} the thread to adjust (this structure is provided by the base code), and thus will enable to run said thread longer (or shorter) on the next cycle.
With this technique we approach the optimal CPU utilization of 85\%  and reach it after a few cycles. 




\section*{Bonus Task: Handle random spikes in thread load}

For the bonus task, we just added a simple logi in the idle thread (\texttt{loop()}) to periodically disrupt the balance by changing the value in \texttt{threadLoad[tid]} and thus changing the load behaviour of one thread. Our load-balancing logic takes care on the next cycle to adjust the values (across all threads) and restore the balance to the target load.
